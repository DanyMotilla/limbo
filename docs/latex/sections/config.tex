\chapter{Configuration Reference}
This appendix provides detailed configuration information for both frontend and backend components.

\section{Frontend Configuration}
\subsection{Environment Variables}
\begin{tcolorbox}[title=.env]
\begin{lstlisting}
# API Configuration
VITE_API_URL=http://localhost:8000
VITE_API_TIMEOUT=30000
VITE_API_RETRY_ATTEMPTS=3

# File Upload
VITE_MAX_FILE_SIZE=50000000
VITE_ALLOWED_EXTENSIONS=.stl

# UI Configuration
VITE_DEFAULT_THEME=light
VITE_ENABLE_ANALYTICS=false
VITE_POLLING_INTERVAL=1000
\end{lstlisting}
\end{tcolorbox}

\subsection{Build Configuration}
\begin{tcolorbox}[title=vite.config.js]
\begin{lstlisting}[language=javascript]
export default {
  server: {
    port: 5173,
    strictPort: true
  },
  build: {
    outDir: 'dist',
    sourcemap: true,
    chunkSizeWarningLimit: 2000
  },
  optimizeDeps: {
    include: ['three', '@react-three/fiber']
  }
}
\end{lstlisting}
\end{tcolorbox}

\section{Backend Configuration}
\subsection{Environment Variables}
\begin{tcolorbox}[title=.env]
\begin{lstlisting}
# Server Configuration
PORT=8000
HOST=0.0.0.0
WORKERS=4
DEBUG=false

# Processing Limits
MAX_VERTICES=1000000
MAX_FILE_SIZE=50000000
MAX_MEMORY_PERCENT=50
MAX_RESOLUTION=100

# Security
CORS_ORIGINS=http://localhost:5173
RATE_LIMIT_PER_MINUTE=10
MAX_CONCURRENT_JOBS=2

# Storage
TEMP_DIR=/tmp/limbo
RESULT_RETENTION_HOURS=24
\end{lstlisting}
\end{tcolorbox}

\subsection{FastAPI Configuration}
\begin{tcolorbox}[title=main.py]
\begin{lstlisting}[language=python]
app = FastAPI(
    title="LIMBO API",
    description="Lattice Interior Mesh Builder & Optimizer",
    version="1.0.0",
    docs_url="/docs",
    redoc_url="/redoc",
    openapi_url="/openapi.json"
)

# CORS configuration
app.add_middleware(
    CORSMiddleware,
    allow_origins=["*"],
    allow_methods=["*"],
    allow_headers=["*"]
)

# Rate limiting
app.add_middleware(
    RateLimitMiddleware,
    calls=10,
    period=60
)
\end{lstlisting}
\end{tcolorbox}

\section{Development Tools}
\subsection{ESLint Configuration}
\begin{tcolorbox}[title=.eslintrc.js]
\begin{lstlisting}[language=javascript]
module.exports = {
  extends: [
    'eslint:recommended',
    'plugin:react/recommended',
    'plugin:@typescript-eslint/recommended'
  ],
  rules: {
    'react/react-in-jsx-scope': 'off',
    'no-unused-vars': 'warn',
    '@typescript-eslint/explicit-module-boundary-types': 'off'
  }
}
\end{lstlisting}
\end{tcolorbox}

\subsection{Python Development Tools}
\begin{tcolorbox}[title=setup.cfg]
\begin{lstlisting}
[flake8]
max-line-length = 88
extend-ignore = E203
exclude = .git,__pycache__,build,dist

[mypy]
python_version = 3.8
warn_return_any = True
warn_unused_configs = True
disallow_untyped_defs = True

[tool:pytest]
testpaths = tests
python_files = test_*.py
addopts = --verbose
\end{lstlisting}
\end{tcolorbox}

\section{Docker Configuration}
\subsection{Development Docker Compose}
\begin{tcolorbox}[title=docker-compose.yml]
\begin{lstlisting}[language=yaml]
version: '3.8'
services:
  frontend:
    build:
      context: .
      dockerfile: Dockerfile.frontend
    ports:
      - "5173:5173"
    volumes:
      - .:/app
      - /app/node_modules
    environment:
      - NODE_ENV=development
      
  backend:
    build:
      context: ./api
      dockerfile: Dockerfile
    ports:
      - "8000:8000"
    volumes:
      - ./api:/app
    environment:
      - PYTHONUNBUFFERED=1
      - DEBUG=true
\end{lstlisting}
\end{tcolorbox}

\section{CI/CD Configuration}
\subsection{GitHub Actions}
\begin{tcolorbox}[title=.github/workflows/ci.yml]
\begin{lstlisting}[language=yaml]
name: CI

on:
  push:
    branches: [ main, develop ]
  pull_request:
    branches: [ main, develop ]

jobs:
  test:
    runs-on: ubuntu-latest
    steps:
      - uses: actions/checkout@v2
      
      - name: Setup Node.js
        uses: actions/setup-node@v2
        with:
          node-version: '16'
          
      - name: Setup Python
        uses: actions/setup-python@v2
        with:
          python-version: '3.8'
          
      - name: Install dependencies
        run: |
          npm install
          cd api && pip install -r requirements.txt
          
      - name: Run tests
        run: |
          npm test
          cd api && pytest
\end{lstlisting}
\end{tcolorbox}
